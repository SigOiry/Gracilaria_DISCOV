% Options for packages loaded elsewhere
\PassOptionsToPackage{unicode}{hyperref}
\PassOptionsToPackage{hyphens}{url}
\PassOptionsToPackage{dvipsnames,svgnames,x11names}{xcolor}
%
\documentclass[
  letterpaper,
  DIV=11,
  numbers=noendperiod]{scrartcl}

\usepackage{amsmath,amssymb}
\usepackage{iftex}
\ifPDFTeX
  \usepackage[T1]{fontenc}
  \usepackage[utf8]{inputenc}
  \usepackage{textcomp} % provide euro and other symbols
\else % if luatex or xetex
  \usepackage{unicode-math}
  \defaultfontfeatures{Scale=MatchLowercase}
  \defaultfontfeatures[\rmfamily]{Ligatures=TeX,Scale=1}
\fi
\usepackage{lmodern}
\ifPDFTeX\else  
    % xetex/luatex font selection
\fi
% Use upquote if available, for straight quotes in verbatim environments
\IfFileExists{upquote.sty}{\usepackage{upquote}}{}
\IfFileExists{microtype.sty}{% use microtype if available
  \usepackage[]{microtype}
  \UseMicrotypeSet[protrusion]{basicmath} % disable protrusion for tt fonts
}{}
\makeatletter
\@ifundefined{KOMAClassName}{% if non-KOMA class
  \IfFileExists{parskip.sty}{%
    \usepackage{parskip}
  }{% else
    \setlength{\parindent}{0pt}
    \setlength{\parskip}{6pt plus 2pt minus 1pt}}
}{% if KOMA class
  \KOMAoptions{parskip=half}}
\makeatother
\usepackage{xcolor}
\setlength{\emergencystretch}{3em} % prevent overfull lines
\setcounter{secnumdepth}{5}
% Make \paragraph and \subparagraph free-standing
\makeatletter
\ifx\paragraph\undefined\else
  \let\oldparagraph\paragraph
  \renewcommand{\paragraph}{
    \@ifstar
      \xxxParagraphStar
      \xxxParagraphNoStar
  }
  \newcommand{\xxxParagraphStar}[1]{\oldparagraph*{#1}\mbox{}}
  \newcommand{\xxxParagraphNoStar}[1]{\oldparagraph{#1}\mbox{}}
\fi
\ifx\subparagraph\undefined\else
  \let\oldsubparagraph\subparagraph
  \renewcommand{\subparagraph}{
    \@ifstar
      \xxxSubParagraphStar
      \xxxSubParagraphNoStar
  }
  \newcommand{\xxxSubParagraphStar}[1]{\oldsubparagraph*{#1}\mbox{}}
  \newcommand{\xxxSubParagraphNoStar}[1]{\oldsubparagraph{#1}\mbox{}}
\fi
\makeatother


\providecommand{\tightlist}{%
  \setlength{\itemsep}{0pt}\setlength{\parskip}{0pt}}\usepackage{longtable,booktabs,array}
\usepackage{calc} % for calculating minipage widths
% Correct order of tables after \paragraph or \subparagraph
\usepackage{etoolbox}
\makeatletter
\patchcmd\longtable{\par}{\if@noskipsec\mbox{}\fi\par}{}{}
\makeatother
% Allow footnotes in longtable head/foot
\IfFileExists{footnotehyper.sty}{\usepackage{footnotehyper}}{\usepackage{footnote}}
\makesavenoteenv{longtable}
\usepackage{graphicx}
\makeatletter
\newsavebox\pandoc@box
\newcommand*\pandocbounded[1]{% scales image to fit in text height/width
  \sbox\pandoc@box{#1}%
  \Gscale@div\@tempa{\textheight}{\dimexpr\ht\pandoc@box+\dp\pandoc@box\relax}%
  \Gscale@div\@tempb{\linewidth}{\wd\pandoc@box}%
  \ifdim\@tempb\p@<\@tempa\p@\let\@tempa\@tempb\fi% select the smaller of both
  \ifdim\@tempa\p@<\p@\scalebox{\@tempa}{\usebox\pandoc@box}%
  \else\usebox{\pandoc@box}%
  \fi%
}
% Set default figure placement to htbp
\def\fps@figure{htbp}
\makeatother
% definitions for citeproc citations
\NewDocumentCommand\citeproctext{}{}
\NewDocumentCommand\citeproc{mm}{%
  \begingroup\def\citeproctext{#2}\cite{#1}\endgroup}
\makeatletter
 % allow citations to break across lines
 \let\@cite@ofmt\@firstofone
 % avoid brackets around text for \cite:
 \def\@biblabel#1{}
 \def\@cite#1#2{{#1\if@tempswa , #2\fi}}
\makeatother
\newlength{\cslhangindent}
\setlength{\cslhangindent}{1.5em}
\newlength{\csllabelwidth}
\setlength{\csllabelwidth}{3em}
\newenvironment{CSLReferences}[2] % #1 hanging-indent, #2 entry-spacing
 {\begin{list}{}{%
  \setlength{\itemindent}{0pt}
  \setlength{\leftmargin}{0pt}
  \setlength{\parsep}{0pt}
  % turn on hanging indent if param 1 is 1
  \ifodd #1
   \setlength{\leftmargin}{\cslhangindent}
   \setlength{\itemindent}{-1\cslhangindent}
  \fi
  % set entry spacing
  \setlength{\itemsep}{#2\baselineskip}}}
 {\end{list}}
\usepackage{calc}
\newcommand{\CSLBlock}[1]{\hfill\break\parbox[t]{\linewidth}{\strut\ignorespaces#1\strut}}
\newcommand{\CSLLeftMargin}[1]{\parbox[t]{\csllabelwidth}{\strut#1\strut}}
\newcommand{\CSLRightInline}[1]{\parbox[t]{\linewidth - \csllabelwidth}{\strut#1\strut}}
\newcommand{\CSLIndent}[1]{\hspace{\cslhangindent}#1}

\KOMAoption{captions}{tableheading}
\makeatletter
\@ifpackageloaded{caption}{}{\usepackage{caption}}
\AtBeginDocument{%
\ifdefined\contentsname
  \renewcommand*\contentsname{Table of contents}
\else
  \newcommand\contentsname{Table of contents}
\fi
\ifdefined\listfigurename
  \renewcommand*\listfigurename{List of Figures}
\else
  \newcommand\listfigurename{List of Figures}
\fi
\ifdefined\listtablename
  \renewcommand*\listtablename{List of Tables}
\else
  \newcommand\listtablename{List of Tables}
\fi
\ifdefined\figurename
  \renewcommand*\figurename{Figure}
\else
  \newcommand\figurename{Figure}
\fi
\ifdefined\tablename
  \renewcommand*\tablename{Table}
\else
  \newcommand\tablename{Table}
\fi
}
\@ifpackageloaded{float}{}{\usepackage{float}}
\floatstyle{ruled}
\@ifundefined{c@chapter}{\newfloat{codelisting}{h}{lop}}{\newfloat{codelisting}{h}{lop}[chapter]}
\floatname{codelisting}{Listing}
\newcommand*\listoflistings{\listof{codelisting}{List of Listings}}
\makeatother
\makeatletter
\makeatother
\makeatletter
\@ifpackageloaded{caption}{}{\usepackage{caption}}
\@ifpackageloaded{subcaption}{}{\usepackage{subcaption}}
\makeatother

\usepackage{bookmark}

\IfFileExists{xurl.sty}{\usepackage{xurl}}{} % add URL line breaks if available
\urlstyle{same} % disable monospaced font for URLs
\hypersetup{
  pdftitle={DISCOV-Gracilaria Paper},
  pdfauthor={Simon Oiry¹; Bede Ffinian Rowe Davies¹; Pierre Gernez¹; Laurent Barillé¹},
  pdfkeywords={Remote Sensing, Invasive species, Coastal
Ecosystems, Biodiversity},
  colorlinks=true,
  linkcolor={blue},
  filecolor={Maroon},
  citecolor={Blue},
  urlcolor={Blue},
  pdfcreator={LaTeX via pandoc}}


\title{DISCOV-Gracilaria Paper}
\author{Simon Oiry¹ \and Bede Ffinian Rowe Davies¹ \and Pierre
Gernez¹ \and Laurent Barillé¹}
\date{2024-11-16}

\begin{document}
\maketitle
\begin{abstract}
To be Written
\end{abstract}


\footnote{Institut des Substances et Organismes de la Mer, ISOMer,
  Nantes Université, UR 2160, F-44000 Nantes, France}

\section{Introduction}\label{introduction}

Coastal ecosystems are among the most dynamic and productive
environments on Earth, providing invaluable ecosystem services and
supporting immense biodiversity. These ecosystems, spanning mangroves,
salt marshes, seagrass meadows, and rocky intertidal zones, play a
pivotal role in carbon sequestration, nutrient cycling, and shoreline
stabilization. They also serve as critical habitats for numerous
species, many of which are commercially or ecologically significant.
Coastal areas are densely populated, with billions of people globally
depending on their resources for livelihoods, fisheries, and tourism.
However, coastal ecosystems face mounting pressures from human
activities such as land reclamation, pollution, and overfishing,
compounded by the impacts of climate change. Sea level rise, ocean
acidification, and increasing storm intensity further exacerbate the
vulnerability of these systems, threatening their resilience and the
services they provide. Protecting and sustainably managing these
ecosystems is therefore critical for maintaining global biodiversity and
supporting human well-being.

One of the significant threats to coastal ecosystems is biological
invasions by non-native species, which can disrupt native biodiversity
and alter ecosystem functions (Krueger-Hadfield (2018) ; Capdevila et
al. (2019) ;Liu et al. (2020)). \emph{Gracilaria vermiculophylla}, an
invasive red macroalga native to the northwest Pacific, exemplifies this
issue. Over the last century, this species has spread extensively across
temperate estuaries in North America, Europe, and other regions,
facilitated by aquaculture and maritime activities (Rueness (2005) ;
Weinberger et al. (2008) ; Krueger-Hadfield et al. (2017)). Its success
as an invader stems from its tolerance to a wide range of environmental
stressors, including temperature (Sotka et al. (2018)), salinity
(Weinberger et al. (2008)), and nutrient variability (Abreu et al.
(2011)), as well as its ability to establish in soft sediment habitats
traditionally devoid of macroalgae (Ramus et al. (2017)). While \emph{G.
vermiculophylla} can provide some ecosystem services, such as habitat
for invertebrates and juvenile fish, it often outcompetes native
vegetation, alters sediment composition (Nyberg et al. (2009)), and
disrupts trophic interactions (Ginneken et al. (2018)). In regions like
the Baltic Sea and the eastern United States, it has been documented to
negatively affect native fucoids and seagrasses (Van Katwijk (2003) ;
Thomsen et al. (2013)). These impacts underscore the importance of
monitoring and managing the spread of \emph{G. vermiculophylla},
particularly as climate change and anthropogenic pressures continue to
facilitate biological invasions.

Remote sensing has revolutionized our ability to monitor and manage
ecosystems, offering efficient and scalable methods for detecting
environmental changes over large areas. Among these technologies,
drone-based remote sensing has emerged as a particularly promising tool
for studying coastal environments. Equipped with high-resolution cameras
and multispectral or hyperspectral sensors, drones can capture
fine-scale spatial and spectral data, enabling researchers to identify
and map vegetation, detect stress in plants, and monitor changes over
time. Unlike traditional satellite imagery, drones provide the
flexibility to operate in overcast conditions, achieve higher spatial
resolution, and target specific areas of interest. For invasive species
like G. vermiculophylla, drones equipped with multispectral sensors can
differentiate it from native vegetation based on its unique spectral
reflectance characteristics. This capability not only enhances detection
accuracy but also reduces the time and labor associated with traditional
field surveys. As the cost of drone technology continues to decrease and
advancements in machine learning facilitate data analysis, drone-based
remote sensing is becoming increasingly accessible and impactful for
ecological research and management.

In this study, we aim to harness the potential of drone-based
multispectral remote sensing to map Gracilaria vermiculophylla in
intertidal zones. Bla bla what are we going to do ? bla bla .

\phantomsection\label{refs}
\begin{CSLReferences}{1}{0}
\bibitem[\citeproctext]{ref-abreu2011nitrogen}
Abreu, M.H., Pereira, R., Buschmann, A., Sousa-Pinto, I., Yarish, C.,
2011. Nitrogen uptake responses of gracilaria vermiculophylla (ohmi)
papenfuss under combined and single addition of nitrate and ammonium.
Journal of Experimental Marine Biology and Ecology 407, 190--199.

\bibitem[\citeproctext]{ref-capdevila2019warming}
Capdevila, P., Hereu, B., Salguero-Gómez, R., Rovira, G. la, Medrano,
A., Cebrian, E., Garrabou, J., Kersting, D.K., Linares, C., 2019.
Warming impacts on early life stages increase the vulnerability and
delay the population recovery of a long-lived habitat-forming macroalga.
Journal of Ecology 107, 1129--1140.

\bibitem[\citeproctext]{ref-van2018global}
Ginneken, V. van, Vries, E. de, others, 2018. The global dispersal of
the non-endemic invasive red alga gracilaria vermiculophylla in the
ecosystems of the euro-asia coastal waters including the wadden sea
unesco world heritage coastal area: Awful or awesome? Oceanography \&
Fisheries Open Access Journal 8, 4--26.

\bibitem[\citeproctext]{ref-krueger2018everywhere}
Krueger-Hadfield, S., 2018. Everywhere you look, everywhere you go,
there's an estuary invaded by the red seaweed gracilaria vermiculophylla
(ohmi) papenfuss, 1967. BioInvasions Records 7.

\bibitem[\citeproctext]{ref-krueger2017genetic}
Krueger-Hadfield, S.A., Kollars, N.M., Strand, A.E., Byers, J.E.,
Shainker, S.J., Terada, R., Greig, T.W., Hammann, M., Murray, D.C.,
Weinberger, F., others, 2017. Genetic identification of source and
likely vector of a widespread marine invader. Ecology and evolution 7,
4432--4447.

\bibitem[\citeproctext]{ref-liu2020ocean}
Liu, C., Zou, D., Liu, Z., Ye, C., 2020. Ocean warming alters the
responses to eutrophication in a commercially farmed seaweed,
gracilariopsis lemaneiformis. Hydrobiologia 847, 879--893.

\bibitem[\citeproctext]{ref-nyberg2009flora}
Nyberg, C.D., Thomsen, M.S., Wallentinus, I., 2009. Flora and fauna
associated with the introduced red alga gracilaria vermiculophylla.
European Journal of Phycology 44, 395--403.

\bibitem[\citeproctext]{ref-ramus2017invasive}
Ramus, A.P., Silliman, B.R., Thomsen, M.S., Long, Z.T., 2017. An
invasive foundation species enhances multifunctionality in a coastal
ecosystem. Proceedings of the national academy of sciences 114,
8580--8585.

\bibitem[\citeproctext]{ref-rueness2005life}
Rueness, J., 2005. Life history and molecular sequences of gracilaria
vermiculophylla (gracilariales, rhodophyta), a new introduction to
european waters. Phycologia 44, 120--128.

\bibitem[\citeproctext]{ref-sotka2018combining}
Sotka, E.E., Baumgardner, A.W., Bippus, P.M., Destombe, C., Duermit,
E.A., Endo, H., Flanagan, B.A., Kamiya, M., Lees, L.E., Murren, C.J.,
others, 2018. Combining niche shift and population genetic analyses
predicts rapid phenotypic evolution during invasion. Evolutionary
Applications 11, 781--793.

\bibitem[\citeproctext]{ref-thomsen2013effects}
Thomsen, M.S., Stæhr, P.A., Nejrup, L., Schiel, D.R., 2013. Effects of
the invasive macroalgae gracilaria vermiculophylla on two co-occurring
foundation species and associated invertebrates. Aquatic Invasions 8,
133--145.

\bibitem[\citeproctext]{ref-van2003reintroduction}
Van Katwijk, M., 2003. Reintroduction of eelgrass (zostera marina l.) in
the dutch wadden sea: A research overview and management vision, in:
Challenges to the Wadden Sea Area. In: Proceedings of the 10th
International Scientific Wadden Sea Symposium, Groningen, the
Netherlands. pp. 173--195.

\bibitem[\citeproctext]{ref-weinberger2008invasive}
Weinberger, F., Buchholz, B., Karez, R., Wahl, M., 2008. The invasive
red alga gracilaria vermiculophylla in the baltic sea: Adaptation to
brackish water may compensate for light limitation. Aquatic Biology 3,
251--264.

\end{CSLReferences}




\end{document}
